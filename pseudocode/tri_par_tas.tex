\documentclass{article}

\usepackage{algorithm}
\usepackage{algpseudocodex}

\begin{document}

\begin{algorithm}
	\begin{algorithmic}[1]
		\caption{Cherche Père (G, D)}
		\State $ Taille \gets D - G + 1 $
		\State $ i \gets Taille / 2 $
		\Return $i$
	\end{algorithmic}
\end{algorithm}

\begin{algorithm}
	\begin{algorithmic}[2]
		\caption{Transpose T en Arbre (T, G, D)}
		\For{$ i \gets G, \dots, D $} step: 3
			\State $ A \gets T[i] $
			\State $ Gauche(A) \gets T[i+1]$ lorsque $T[i+1]$ existe
			\State $ Droite(A) \gets T[i+2]$ lorsque $T[i+2]$ existe
			\State $ A \gets Suivant(A) $
		\EndFor
		\Return $A$
	\end{algorithmic}
\end{algorithm}

\begin{algorithm}
	\begin{algorithmic}[3]
		\caption{Tri Par Tas (T, G, D)}
		\State $ P \gets \Call{Cherche Pere}{G, D} $
		\State $ A \gets \Call{Transpose T en Arbre}{T, G, D} $
		\For{$ i = P, \dots, G $}
			\State $ A \gets Indice(A, i) $
			\State $ i \gets i - 3 $
			\State $ \Call{Rendre Maximier}{T, i, G + P - i + 1} $
		\EndFor
	\end{algorithmic}
\end{algorithm}

\end{document}
